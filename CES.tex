\documentclass[UTF8]{ctexart}
\usepackage[a4paper, margin=1in]{geometry} % 设置页面大小和边距
\usepackage{amsmath, amssymb, amsthm}     % 数学公式支持
\usepackage{hyperref}                     % 超链接
\usepackage{graphicx}                     % 插入图片
\usepackage{enumitem}                     % 自定义列表格式
\usepackage{fancyhdr}                     % 页眉页脚
\usepackage{titlesec}                     % 控制标题样式
\usepackage{setspace}                     % 控制行距
\usepackage{xcolor}                       % 颜色支持


\titleformat{\section}
  {\normalfont\Huge\bfseries}{\thesection}{1em}{}

\titleformat{\subsection}
  {\normalfont\Large\bfseries}{\thesubsection}{1em}{}

\titleformat{\subsubsection}
  {\normalfont\large\bfseries}{\thesubsubsection}{1em}{}
% 定理环境定义
\newtheorem{definition}{Definition}[section]
\newtheorem{theorem}{Theorem}[definition]
\newtheorem{lemma}[theorem]{Lemma} 
\newtheorem{corollary}[theorem]{Corollary}
\theoremstyle{definition}

\newtheorem{example}{\kaishu 例}[section]
\newtheorem{remark}{Remark}[section]
% 页眉页脚设置
\pagestyle{fancy}
\fancyhead[L]{SOE Workshop}
\fancyhead[R]{}
\fancyfoot[C]{\thepage}

% 标题样式设置
\titleformat{\section}{\large\bfseries}{\thesection.}{1em}{}
\titleformat{\subsection}{\normalsize\bfseries}{\thesubsection.}{1em}{}
\definecolor{myLinkColor}{RGB}{0,0,0} % Dark blue for links
\definecolor{myCiteColor}{RGB}{180,0,0}  % Dark red for URLs
\definecolor{myUrlColor}{RGB}{0,0,180}  % Dark red for URLs

% Configure hyperlink appearance
\hypersetup{
    colorlinks=true,       % false: boxed links; true: colored links
    linkcolor=myLinkColor, % Color of internal links (sections, equations)
    citecolor=myCiteColor, % Color of citations
    urlcolor=myUrlColor,   % Color of URLs
    linkbordercolor={1 0 0}, % Color of link borders if colorlinks=false
}

% 开始文档
\begin{document}

\title{Introduction to CES Functions}
\author{曹与时\footnote{
  复旦大学经济学院。电子信箱: yscao22@m.fudan.edu.cn.
}}
\date{}
\maketitle

\begin{abstract}
这份文档简要介绍了CES(constant elasticity of substitution)函数的基本性质。我们会先介绍弹性、替代弹性的概念,再开始正题。我们会从两商品的CES效用函数出发,逐步扩展到$n$个离散商品,最后拓展到$n$维连续商品。最终我们会列举一些在经济学理论中涉及到CES函数的模型。
\end{abstract}

\tableofcontents

\newpage
\section{弹性(Elasticity)}
经济学入门中,最令人头疼的概念可以说是弹性了。为什么经济学关心弹性?到底如何理解弹性?我们可以先从经济学原理一定会遇到的例子入手:需求弹性。

\begin{example}[\kaishu 需求弹性]
\kaishu 烤冷面摊位面临一个价格决策的难题。由于“猪周期”波动,火腿肠的成本上升。摊位老板想了解,如果把火腿肠从5元提高到6元,来购买的学生会少多少?老板已经算出来,市场的价格-需求函数是$Q_D(P)=250-10P.$
\end{example}
这里的计算非常容易。
\[
Q_D(5)=250-10*5=200\text{(人)}, Q_D(6)=250-10*6=190\text{(人)}. 
\]
$\Delta Q=10.$也就是说,学生会少10人。

当然,注意力强的朋友已经注意到了,我们可以利用{\kaishu 导数(差分)}来做这道题。假设(元)是最小单位,那么对于极小变动(5元到6元)而言,我们可以近似地写为:
\[\Delta Q=Q_D'(P) \Delta P |_{P=5}=-10.\text{(人)}.\]
这里,我们把微分$\frac{dQ}{dP}$引入了我们的分析中。这里的$\frac{dQ}{dP}$,单位是:(人/元)。

所以这位老板会损失10位客人。或者说,老板会少$\frac{10}{200}=5\%$的顾客。

这两种表述有什么不一样?第一种表述中,我们算出来的数字是有单位的。而计算百分比时,由于分子分母均有单位,所以单位抵消,我们只剩下数字。

然而,这似乎只是表述上的不同,是否有单位看起来并不重要。那么让我们看下一个例子。
\begin{example}[\kaishu 需求弹性:变式1]
\kaishu 假设这个世界上不存在单身狗,只有小情侣。摊位老板想了解,如果把火腿肠从5元提高到6元,来购买的小情侣会少多少对?注意到,市场的价格-需求函数会发生改变,成为$Q_D(P)=125-5P.$
\[
Q_D(5)=125-5*5=100\text{(对)}, Q_D(6)=125-5*6=95\text{(对)}. 
\]
\end{example}
$\Delta Q=5.$也就是说,老板会损失5对小情侣顾客。也即,老板会损失$\frac{5}{100}=5\%$的顾客。

和{\kaishu{例1.1}}相比,我们发现:存在单位的结果会随着单位而变化,而比例则展现了其稳健性。事实上两个例子在现实生活中并没有任何差别(当然第二个例子不太现实)。经济学研究中,量纲会变得非常复杂。为了排除量纲带来的影响,我们选取比例作为研究对象。

\begin{example}[\kaishu 需求弹性:变式2]
\kaishu 现在重新以人为单位。烤冷面摊位决定逃离小破五角场,来到了徐家汇。现在烤肠的基本价是24元(感觉可以更高)。摊位老板想了解,如果把火腿肠从24元提高到25元,来购买的顾客会少多少?假设所有复旦学生回家都路过徐家汇,也即:市场的价格-需求函数仍然与例1.1一样, 是$Q_D(P)=250-10P.$
\[
Q_D(5)=250-10*24=10\text{(人)}, Q_D(6)=250-10*25=0\text{(人)}. 
\]
\end{example}
$\Delta Q=10.$由于价格需求函数是线性的,这里不会有任何变化。所以10(人)这一点,和之前{\kaishu{例1.1}}的计算是一样的。但是老板损失了多少比例的顾客?$\frac{10}{10}=100\%$!也即,老板提了同样单位的价格,却损失了所有顾客。

我们可以清楚的看到{\kaishu{例1.1}}和{\kaishu{例1.3}}的差异:由于基础价格的不同,即使老板损失的顾客数量是一样的,损失顾客的比例天差地别。也即:损失顾客的比例具有\textit{价格效应}。 

在通常的经济学建模中,除非我们关注价格效应(或者价格带来的规模效应),我们一般会设定其为\textit{恒定}的。

由此,我们正式引入弹性的定义。


\begin{definition}[弹性]
如果变量y是x的函数,即y=y(x),那么y对x的弹性可以定义为:
\[\epsilon=e_{yx}=\frac{\frac{\Delta y}{y}}{\frac{\Delta x}{x}}=\frac{d ln y}{d ln x}=\frac{\Delta y}{\Delta x} \frac{x}{y}=\frac{d y(x)}{dx}\frac{x}{y}.\]
\end{definition}
由定义$e_{yx}=\frac{\frac{\Delta y}{y}}{\frac{\Delta x}{x}}$可以看出,弹性衡量的是:因变量y由自变量x比例上的变动($\frac{\Delta x}{x}$)而产生的比例上的变动($\frac{\Delta y}{y}$)。 不但如此,更准确地来说,弹性衡量的是这种\textit{比例变动的速度}。

在我们之前的例子中,我们可以分别计算$P=5$和$P=24$对应的\textit{点弹性}。
\[\epsilon|_{P=5}=\frac{d Q_D(P)}{dP}\frac{P}{Q}=-10*\frac{5}{200}=-0.25, \]
\[\epsilon|_{P=24}=\frac{d Q_D(P)}{dP}\frac{P}{Q}=-10*\frac{24}{10}=-24.\]
很明显,需求弹性在不同价格下是不一样的,也即存在价格的规模效应。

\section{边际替代率(Marginal Rate of Substitution)}
在过渡到替代弹性之前,我们先介绍一下什么是边际替代率(MRS)。我们举一个例子。
\begin{example}[两商品的效用函数]
假设我们的效用函数服从C-D形式:$U(c_1,c_2)=c_1^\alpha c_2^{1-\alpha}$。给定现在消费$\bar c_1$单位1商品和$\bar c_2$单位2商品,带来$U_1$的效用水平。
\end{example}
此时边际替代率是什么?是给定效用水平$U_1$不变的时候,增加1单位的商品1的消费会导致商品2消费减少的量。很抽象,是不是?

我们来进一步深究里面的经济机制。如果效用水平不变的话,agent选择1个商品1还是选择x个商品2,应该是\textit{无差异}的。换句话而言,1个商品1带来的边际效用,与x个商品2带来的边际效用,应该相同的。也即:
\[\frac{\partial U}{\partial c_1}|_{c_1=\bar c_1,c_2=\bar c_2}=x \frac{\partial U}{\partial c_2}|_{c_1=\bar c_1, c_2=\bar c_2}.\]
此处的$x$正是我们要找的边际替代率的绝对值。由于在我们现在的设定下,商品1消费的增加带来的是商品2消费的减少,所以$MRS=-x$.


\begin{definition}[边际替代率]
给定消费组合$(x,y)$对应的效用无差异曲线$U$,那么在无差异曲线$U_1$上,商品$x$对商品$y$的边际替代率可以定义为
    \[MRS=\frac{dy(x)}{dx}|_{U=U_1}=-\frac{MU_1}{MU_2}|_{U=U_1}.\]
\end{definition}
其中$MU_1$就是商品1的边际效益,也即$\frac{\partial U}{\partial c_1}$。一单位的$x$商品消费的增加带来$MRS$单位$y$商品的消费。
\begin{remark}
也有教科书为了简洁起见,将$MRS$直接定义为$\frac{MU_1}{MU_2}|_{U=U_1}$。在此,我们以范里安的定义为基准。   
\end{remark}

\begin{example}[C-D函数的边际替代率]
现在开始计算给定C-D函数的边际替代率。
\[MU_1=\frac{\partial U}{\partial c_1}=\alpha c_1^{\alpha-1}c_2^{1-\alpha}|_{c_1=\bar c_1,c_2=\bar c_2}=\alpha(\frac{\bar c_2}{\bar c_1})^{1-\alpha}.\]
\[MU_2=\frac{\partial U}{\partial c_2}=(1-\alpha)c_1^{\alpha}c_2^{-\alpha}|_{c_1=\bar c_1,c_2=\bar c_2}=(1-\alpha)(\frac{\bar c_1}{\bar c_2})^{\alpha}.\]
\[MRS=-\frac{MU_1}{MU_2}|_{U=U_1}=-\frac{\alpha}{1-\alpha}(\frac{\bar c_2}{\bar c_1}).\]
在此处,我们可以看到,C-D函数的边际替代率仅与两商品消费的比例有关,与消费的总规模无关。换句话而言,C-D函数的边际替代率不存在规模效应。
\end{example}

\section{边际技术替代率(Marginal rate of technical substitution/Technical rate of substitution)}
上一章我们介绍的是对应效用曲线的\textit{边际替代率(MRS)}。对于生产函数,我们有类似的定义,此时将商品1和2换成了资本和劳动:

\begin{definition}[边际技术替代率]
给定要素投入组合$(k,l)$对应的产出无差异曲线$U$,那么在无差异曲线$U_1$上,劳动$l$对资本$k$的边际技术替代率可以定义为
    \[MRTS=\frac{dk}{dl}|_{Y=Y_1}=-\frac{MPL}{MPK}|_{Y=Y_1}.\]
\end{definition}
其中$MPK$就是资本的边际产出,也即$\frac{\partial Y}{\partial k}$。

\section{替代弹性(Elasticity of Substitution)}
有了弹性和边际替代率的基础,我们来介绍替代弹性。需求弹性衡量的是:一单位价格变动的比例带来的需求变动的比例,那么依葫芦画瓢,替代弹性是什么?是单位\textit{边际替代率}变化的比例带来的\textit{替代}变动的比例。

什么是\textit{替代}变动的比例?\textit{替代}指的是消费品组合中,商品1的消费被商品2的消费所替代。也就是说,我们关心的变量是$\frac{c_2}{c_1}$。由此,我们可以定义替代弹性:
\begin{definition}[替代弹性]
对于消费者而言,假定有商品1和商品2,那么替代弹性可以被定义为:
\[\sigma=\frac{\frac{\Delta\frac{c_2}{c_1}}{\frac{c_2}{c_1}}}{\frac{\Delta |MRS|}{|MRS|}}=\frac{d\ln \frac{c_2}{c_1}}{d\ln |MRS|}.\]
对于生产者而言,替代弹性可以被定义为:
\[\sigma=\frac{\frac{\Delta\frac{k}{l}}{\frac{k}{l}}}{\frac{\Delta |MRTS|}{|MRTS|}}=\frac{d\ln \frac{k}{l}}{d\ln |MRTS|}.\]
\end{definition}
为何我们关心常替代弹性(constant elasticity of substitution)?正如烤肠的例子所示,如果替代弹性不是常数,就会出现(价格上的)规模效应。然而,在经济建模中,大部分时间,我们不希望替代弹性会随着消费规模的扩大而发生变化————只要要素之间比例相同,他们的替代弹性就应当是一样的。否则,模型会变得非常复杂:不同规模的厂商/消费者有不同的行为。当然,在更贴近现实的建模(Heterogeneous Agent Model)中,这一点假设可以放松。

\begin{example}[C-D函数的替代弹性]
\kaishu 证明C-D函数的替代弹性是常数,也即:C-D函数是一类CES函数。
\[\sigma=\frac{d\ln \frac{c_2}{c_1}}{d\ln |MRS|}=\frac{d\ln \frac{c_2}{c_1}}{d\ln\frac{\alpha}{1-\alpha}(\frac{c_2}{c_1})}=\left[\frac{d\ln\frac{\alpha}{1-\alpha}+d\ln(\frac{c_2}{c_1})}{d\ln \frac{c_2}{c_1}}\right]^{-1}=1.\]
\end{example}

\section{两商品的CES函数与CES生产函数}
在定义完替代弹性后,我们终于进入正题:CES函数。首先我们来看一个最简单的情况:两个商品的CES函数。

\begin{definition}[两商品的CES效用函数]
对于两商品的CES效用函数,我们定义为
\[U(x,y)=\left[a_1 x^\rho+a_2 y^\rho\right]^\frac{1}{\rho},\rho\leq 1.\]
\end{definition}
类似地,我们可以定义CES生产函数。
\begin{definition}[CES生产函数]
对于CES生产函数,我们定义为
\[U(x,y)=A\left[a K^\rho+(1-a) L^\rho\right]^\frac{1}{\rho},\rho\leq 1.\]
\end{definition}
\begin{theorem}[替代弹性]
对于CES形式的效用函数/生产函数,替代弹性为
\[\sigma=\frac{1}{1-\rho}.\]
\end{theorem}
\begin{theorem}[Leontief 生产函数]
当替代弹性$\sigma\rightarrow0$, $\rho\rightarrow-\infty$, 函数形式演变为$Y = A\min[K, L]$.
\end{theorem}
\begin{theorem}[Linear 生产函数]
当替代弹性$\sigma\rightarrow +\infty$, $\rho\rightarrow1$, 函数形式演变为$Y = A[aK+(1-a)L]$.
\end{theorem}
\begin{theorem}[Cobb-Douglas 生产函数]
当替代弹性$\sigma\rightarrow 1$,$\rho\rightarrow 0$, 函数形式演变为$Y = \Tilde{A} K^a L^{1-a}$.
\end{theorem}

\section{n维离散商品的CES函数}
\begin{definition}[n维离散商品的CES函数]
对于n维离散商品的CES函数,我们定义为
\[U(x)=\left[\sum_i a_i xi^\rho\right]^\frac{1}{\rho},\rho\leq 1.\]
\end{definition}

\section{连续商品的CES函数}
\begin{definition}[连续商品的CES函数]
对于连续商品的CES函数,我们定义为
\[C \equiv\left(\int_{0}^{N} c_{i}^{\frac{\epsilon-1}{\epsilon}} d i\right)^{\epsilon /(\epsilon-1)}\]
\end{definition}



\section{应用:The Armington Model}

\section{应用:垄断竞争——Dixit-Stigliz Model}

\section{拓展:非齐次CES函数与结构转型:Comin et al:(2021, \textit{Econometrica}).}

\section{拓展:非希克斯中性的CES生产函数:Acemoglu\& Restrepo:(2019,\textit{AER}).}

\section{拓展:嵌套CES(Nested CES): Boppart et al:(2023, \textit{Revision for JPE}).}
%\bibliographystyle{plain}
%\bibliography{references} % 引用文献(需创建 references.bib 文件)
\end{document}